\chapter{ Getting information on selected structural features: g(r) in $g$-SiO$_2$}
\label{tuto-1}

The following example illustrates how \atomes\ can be use to retrieve information on selected structural features:

\section{Open the \atomes\ $g$-SiO$_2$ project file}

Open the \atomes\ project file for the system "\href{http://atomes.ipcms.fr/wp-content/uploads/2022/08/gSiO2-custom-rings.tar.bz2}{$g$-SiO$_2$ with ring statistics and custom color map, 3000 atoms}" \\[0.25cm]
\image{14}{img/tuto-1/gSiO2-001}\\
Note that the "\texttt{Toolboxes}" dialog is active and some results, from calculation performed previously and stored in the project file, 
are available, including for the radial distribution function g(r). 

\clearpage

\section{Open the "Environment configuration" window}

Go back to the 3D window of the $g$-SiO$_2$ model, 
for clarity purpose change color map to standard chemical species: from the "\texttt{OpenGL~->~Color~Scheme(s)}" menu, or press \keystroke{a} as many times as required. 
Then open the "\texttt{Environments configuration}" window: from the "\texttt{Chemistry}" menu or press \Ctrl+\keystroke{e}, and open the tab labelled "\texttt{King's~ring(s){\bf{[KR]}}}". \\[0.5cm]
\begin{tabular}{cp{0.5cm}c}
\hspace{-1cm}\image{9}{img/tuto-1/map-env} & \raisebox{2.0cm}{$\Longrightarrow$} & \image{5}{img/tuto-1/gSiO2-003}
\end{tabular} \\[0.5cm]
This tab provides interactors with King's ring statistics calculation results: color of the ring's polyhedra, and then options to show/hide, label/unlabel, 
or select/unselect atom(s) involved in any ring(s) of the corresponding size.

\clearpage

\section{Select all atoms in ring(s) of size 6}

"\texttt{Pick}" all atoms in ring(s) of size 6: atoms in the 3D window are then selected and highlighted in light blue color. \\[0.5cm]
\image{15}{img/tuto-1/sel}

\clearpage

\section{Create a new model using the atom selection} 

Go back to the 3D window, and mouse right click on any selected atom, then browse the contextual menu that pops up to "\texttt{Edit~as~New~Project}", 
then select "\texttt{All~Selected~Atom(s)/Bond(s)}" \\[0.5cm]
\begin{tabular}{cp{0.5cm}c}
\hspace{-1cm}\image{8}{img/tuto-1/edit\_as\_new} & \raisebox{2.0cm}{$\Longrightarrow$} & \image{8}{img/tuto-1/new\_proj}
\end{tabular}
\\[0.5cm]
This will create a new \atomes\ project that will immediately be inserted in the program's workspace, and the corresponding 3D window will appear. \\ 
This new model contains all atoms involved in King's rings of size 6 selected previously in the disordered SiO$_2$ model. 

\clearpage

\section{Correct the periodicity of the new model} 

From the main interface open the "\texttt{Edit}" menu and select "\texttt{Box~and~Periodicity}" 
to reproduce the periodicity of the initial model describing properly the simulation box (a~$=$~b~$=$~c~$= 35.6621$~\AA, $\alpha = \beta = \gamma = 90~^{\circ}$, and be sure to use Periodic Boundary Conditions), then apply changes. \\[0.5cm]
\image{15}{img/tuto-1/perio} \\[0.5cm]
The new model now uses periodic boundary conditions, if required the box can be displayed in the 3D window using the corresponding menu button: "\texttt{Model~->~Box~->~Show/Hide}".

\clearpage

\section{Compute the g(r) for the new model}

Again from the main interface open the "\texttt{Analyze}" menu, then select "\texttt{g(r)~/~G(r)}", to open the associated calculation dialog. 
Set parameter(s) to acceptable value(s) and run the analysis. \\[0.5cm] 
\image{15}{img/tuto-1/run-gr}

\clearpage

\section{Visualize the results of the calculation} 

When the calculation is completed, close the dialog, and note that the content of the "\texttt{Toolboxes}" dialog 
is updated, the results of the calculation being available:\\[0.5cm]
\begin{tabular}{cp{0.5cm}cp{0.5cm}c}
\hspace{-1cm}\image{5}{img/tuto-1/tb-001} & \raisebox{2.0cm}{$\Longrightarrow$} &
\image{5}{img/tuto-1/tb-002} & \raisebox{2.0cm}{$\Longrightarrow$} &
\image{5}{img/tuto-1/tb-003} \\
\end{tabular}
\begin{tabular}{cc}
\hspace{2cm}\image{10}{img/tuto-1/gr} & \raisebox{8cm}{$\Swarrow$}
\end{tabular}
\\[0.5cm]
In the "\texttt{Toolboxes}" dialog open the result of the calculation to display the g(r) of the active project, in this case the new model created from the atom selection. 

\clearpage

\section{Add the calculation data for the entire $g$-SiO$_2$ model}

On the curve window press the mouse right click, browse the menu to "\texttt{Add~Data~Set~->~$g$SiO}$_2$\texttt{~->~g(r)~neutrons}" 
to add the corresponding data set to the representation: \\[0.5cm]
\begin{tabular}{cp{0.5cm}c}
\hspace{-1cm}\jmage{6}{img/tuto-1/add\_data} & \raisebox{3.0cm}{$\Longrightarrow$} & \jmage{7}{img/tuto-1/final}
\end{tabular}
\\[0.5cm]
Note that projects opened in \atomes\ share data. Therefore calculation results, from any model in the workspace and provided that the calculations have been performed, can be compared easily. 
In this example it allows to compare the g(r) for the $g$-SiO$_2$ system with a partial contribution from the atoms involved in King's rings of size 6. \\
Comprehensive layout options are available and presentation can be clarified, each data set customized, so that the results of the analysis are quickly publication ready.
