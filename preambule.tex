\usepackage[utf8]{inputenc}
\usepackage[T1]{fontenc}
\usepackage{babel}

\usepackage{pifont}
% Les polices possibles (pour tout le document)
%\usepackage[bitstream-charter]{mathdesign}

%Font ps pour le pdf
\usepackage{pslatex}

%Hyperliens pour le pdf
\usepackage[pdfauthor   = {Sébastien\ Le\ Roux},
    pdftitle    = {atomes\ tutorials},
    pdfsubject = {atomes\ tutorials},
    pdfkeywords = {atomes\ tutorials\ guide\ help\ documentation},
    pdfcreator = {Latex+DviPDF},
    pdfproducer = {LaTeX+DviPDF},
    pdfstartview=FitV   % Ouverture avec ajustement de l'image
    dvips=true,         % Use hyperref with dvips
    colorlinks=true,    % Lien hypertext en couleur
    plainpages=false,   %
    pagebackref=true,   % Permet d'ajouter des liens retour dans la biblio ...
    backref=page,       % .. ces liens pointent vers les 'pages' des citations
    hyperindex=true,    % Ajoute des liens dans l'index
    linktocpage=true,   % Lien sur les numéros de page et non le text 
    breaklinks=true,    % Permet le retour à la ligne dans les liens trop longs
    urlcolor= blue,     % Couleur des liens externes
%    linkcolor= black,   % Couleur des liens internes
    bookmarks=true,     % Création des signets pour Acrobat
    bookmarksopen=false % Toute l'arborescence est dépliée à l'ouverture
]{hyperref}
%\usepackage[pageref]{backref}

%Insertion d'images
\usepackage{graphicx}
\DeclareGraphicsExtensions{.eps}

%Dessins latex
% Brouillon ? = afficher les images "false" ou seulement les cadres "true"
% effet sur tout le document
\newcommand{\ddst}{false}
%\usepackage{picins}

\usepackage{color}

% Mode verbatim avancé
\usepackage{alltt}

% Système de liste/énumération
\usepackage{pifont}
\usepackage{enumerate}

% Ecriture des mathématiques
\usepackage{amsmath}
\usepackage{amssymb}
\usepackage{amscd}
\usepackage{theorem}

\usepackage{pxfonts}

% tableaux
\usepackage{hhline}
%\usepackage{array}
\usepackage{multirow}
\usepackage{tabls}

%% Mise en page
\voffset         0.0cm
\hoffset         0.0cm
\textheight     24.0cm
\textwidth      16.0cm
\topmargin      -0.5cm
\oddsidemargin   0.0cm
\evensidemargin  0.0cm

% pages en landscape dans document portrait
\usepackage{lscape}

% Aspect de pages
\usepackage{setspace}
%\onehalfspacing
%\doublespacing
%\setstretch{3}

%\usepackage{fancyhdr,fancybox}
%\fancyhead{}
%\fancyhead[RO]{\scriptsize{\slshape\rightmark}}
%\fancyhead[LO]{\scriptsize{\thepage}}
%\fancyhead[RE]{\scriptsize{\thepage}}
%\fancyhead[LE]{\scriptsize{\slshape\leftmark}}
%\pagestyle{fancy}

% Numérotation pour les sous-sous-sections
\setcounter{secnumdepth}{3}
% Table des matières/figures/tables par chapitre
%\usepackage{minitoc}
% Pour placer des notes de bas de pages dans les titres
%\usepackage[stable]{footmisc}

%Définitions de théorèmes
\theoremstyle{plain}
\theoremheaderfont{\scshape}
\theorembodyfont{\normalfont\itshape}
\newtheorem{HK}{Théorème}

% Créer un environnement résumé
\def\abstract{
   \begin{center}
   \begin{minipage}{12cm}
   \begin{center}{\bf Résumé}\end{center}\par\small}
\def\endabstract{\par\end{minipage}\end{center}\vspace{1cm}}

% Bilblio:
% Bilblio:
\newcommand{\aap}{Astron. \& Astrophys.}
\newcommand{\aasup}{Astron. \& Astrophys. Suppl. Ser.}
\newcommand{\aj}{Astron. J.}
\newcommand{\aph}{Acta Phys.}
\newcommand{\act}{Acta Cryst.}
\newcommand{\actc}{Acta Cryst.}
\newcommand{\acta}{Acta Cryst. A}
\newcommand{\actb}{Acta Cryst. B}
\newcommand{\advp}{Adv. Phys.}
\newcommand{\ajp}{Amer. J. Phys.}
\newcommand{\ajm}{Amer. J. Math.}
\newcommand{\amsci}{Amer. Sci.}
\newcommand{\anofd}{Ann. Fluid Dyn.}
\newcommand{\am}{Ann. Math.}
\newcommand{\ap}{Ann. Phys. (NY)}
\newcommand{\adp}{Ann. Phys. (Leipzig)}
\newcommand{\ao}{Appl. Opt.}
\newcommand{\apl}{Appl. Phys. Lett.}
\newcommand{\app}{Astroparticle Phys.}
\newcommand{\apj}{Astrophys. J.}
\newcommand{\apjsup}{Astrophys. J. Suppl.}
\newcommand{\apss}{Astrophys. Space Sci.}
\newcommand{\araa}{Ann. Rev. Astron. Astrophys.}
\newcommand{\baas}{Bull. Amer. Astron. Soc.}
\newcommand{\baps}{Bull. Amer. Phys. Soc.}
\newcommand{\cmp}{Comm. Math. Phys.}
\newcommand{\cpam}{Commun. Pure Appl. Math.}
\newcommand{\cppcf}{Comm. Plasma Phys. \& Controlled Fusion}
\newcommand{\cpc}{Comp. Phys. Comm.}
\newcommand{\cms}{Comp. Mat. Sci.}
\newcommand{\cqg}{Class. Quant. Grav.}
\newcommand{\cra}{C. R. Acad. Sci. A}
\newcommand{\crv}{Chem. Rev.}
\newcommand{\cp}{Chem. Phys.}
\newcommand{\cma}{Chem. Mater.}
\newcommand{\epl}{Eur. Phys. Lett.}
\newcommand{\ejm}{Eur. J. Mineral.}
\newcommand{\fed}{Fusion Eng. \& Design}
\newcommand{\ft}{Fusion Tech.}
\newcommand{\grg}{Gen. Relativ. Gravit.}
\newcommand{\ieeens}{IEEE Trans. Nucl. Sci.}
\newcommand{\ieeeps}{IEEE Trans. Plasma Sci.}
\newcommand{\ijimw}{Interntl. J. Infrared \& Millimeter Waves}
\newcommand{\ip}{Infrared Phys.}
\newcommand{\irp}{Infrared Phys.}
\newcommand{\jap}{J. Appl. Phys.}
\newcommand{\jac}{J. Appl. Cryst.}
\newcommand{\jacs}{J. Am. Chem. Soc.}
\newcommand{\jamcs}{J. Am. Ceram. Soc.}
\newcommand{\jasa}{J. Acoust. Soc. America}
\newcommand{\jcp}{J. Chem. Phys.}
\newcommand{\jcop}{J. Comp. Phys.}
\newcommand{\jetp}{Sov. Phys.--JETP}
\newcommand{\jetpl}{JETP Lett.}
\newcommand{\jfe}{J. Fusion Energy}
\newcommand{\jfm}{J. Fluid Mech.}
\newcommand{\jmp}{J. Math. Phys.}
\newcommand{\jne}{J. Nucl. Energy}
\newcommand{\jnec}{J. Nucl. Energy, C: Plasma Phys., Accelerators, Thermonucl. Res.}
\newcommand{\jncs}{J. Non-Cryst. Solids.}
\newcommand{\jnm}{J. Nucl. Mat.}
\newcommand{\joam}{J. Optoelect. Adv. Mat.}
\newcommand{\jpcssp}{J. Phys. C: Solid State Phys.}
\newcommand{\jpc}{J. Phys. Chem.}
\newcommand{\jpcs}{J. Phys. Chem. Sol.}
\newcommand{\jpcm}{J. Phys.: Cond. Mat.}
\newcommand{\jpp}{J. Plasma Phys.}
\newcommand{\jpsj}{J. Phys. Soc. Japan}
\newcommand{\jqc}{J. Quant. Chem.}
\newcommand{\jssc}{J. Sol. Stat. Chem.}
\newcommand{\jsi}{J. Sci. Instrum.}
\newcommand{\jvst}{J. Vac. Sci. \& Tech.}
\newcommand{\mcp}{Mat. Chem. Phys.}
\newcommand{\nat}{Nature}
\newcommand{\nature}{Nature}
\newcommand{\nedf}{Nucl. Eng. \& Design/Fusion}
\newcommand{\nf}{Nucl. Fusion}
\newcommand{\nim}{Nucl. Inst. \& Meth.}
\newcommand{\nimpr}{Nucl. Inst. \& Meth. in Phys. Res.}
\newcommand{\np}{Nucl. Phys.}
\newcommand{\npb}{Nucl. Phys. B}
\newcommand{\ntf}{Nucl. Tech./Fusion}
\newcommand{\npbpc}{Nucl. Phys. B (Proc. Suppl.)}
\newcommand{\inc}{Nuovo Cimento}
\newcommand{\nc}{Nuovo Cimento}
\newcommand{\pcg}{Phys. Chem. Glasses}
\newcommand{\pf}{Phys. Fluids}
\newcommand{\pfa}{Phys. Fluids A: Fluid Dyn.}
\newcommand{\pfb}{Phys. Fluids B: Plasma Phys.}
\newcommand{\pl}{Phys. Lett.}
\newcommand{\pla}{Phys. Lett. A}
\newcommand{\plb}{Phys. Lett. B}
\newcommand{\prep}{Phys. Rep.}
\newcommand{\pnas}{Proc. Nat. Acad. Sci. USA}
\newcommand{\pp}{Phys. Plasmas}
\newcommand{\ppcf}{Plasma Phys. \& Controlled Fusion}
\newcommand{\phitrsl}{Philos. Trans. Roy. Soc. London}
\newcommand{\plmb}{Phil. Mag. B} 
\newcommand{\pml}{Phil. Mag. Lett.}
\newcommand{\pmm}{Phil. Mag.}
\newcommand{\prl}{Phys. Rev. Lett.}
\newcommand{\pr}{Phys. Rev.}
\newcommand{\physrev}{Phys. Rev.}
\newcommand{\pra}{Phys. Rev. A}
\newcommand{\prb}{Phys. Rev. B}
\newcommand{\prc}{Phys. Rev. C}
\newcommand{\prd}{Phys. Rev. D}
\newcommand{\pre}{Phys. Rev. E}
\newcommand{\ps}{Phys. Scripta}
\newcommand{\pstb}{Phys. Stat. Sol. b}
\newcommand{\procrsl}{Proc. Roy. Soc. London}
\newcommand{\rmp}{Rev. Mod. Phys.}
\newcommand{\rsi}{Rev. Sci. Inst.}
\newcommand{\rpp}{Rep. Prog. Phys.}
\newcommand{\science}{Science}
\newcommand{\sciam}{Sci. Am.}
\newcommand{\susc}{Surf. Sci}
\newcommand{\sam}{Stud. Appl. Math.}
\newcommand{\sjpp}{Sov. J. Plasma Phys.}
\newcommand{\spd}{Sov. Phys.--Doklady}
\newcommand{\sptp}{Sov. Phys.--Tech. Phys.}
\newcommand{\spu}{Sov. Phys.--Uspeki}
\newcommand{\skt}{Sky and Telesc.}
\newcommand{\ssi}{Solid State Ionics}
\newcommand{\ssc}{Solid State Com.}
\newcommand{\ssnmr}{Solid State Nuc. Mag. Res.}
\newcommand{\zfp}{Zs. f. Phys.}
\newcommand{\zk}{Z. Kristallogr.}


%\usepackage{chapterbib}

%\usepackage[square,comma,sort&compress]{natbib}
%\usepackage{hypernat}

% Algo XML
\usepackage{listings}

% Texte souligné
\usepackage[normalem]{ulem}

% Mise en forme des légendes
\usepackage[hang]{caption2}
%\usepackage[hang]{caption}
\renewcommand{\captionfont}{\it}
\renewcommand{\captionlabelfont}{\bf}
\renewcommand{\captionlabeldelim}{$\quad$}

%\usepackage[format=plain,labelfont=bf,up,textfont=it,up]{caption}
\newcommand{\red}[1]{\textcolor{red}{#1}}
\newcommand{\blue}[1]{\textcolor{blue}{#1}}
\newcommand{\green}[1]{\textcolor{green}{#1}}
\newcommand{\violet}[1]{\textcolor{violet}{#1}}
\newcommand{\pink}[1]{\textcolor{pink}{#1}}
\newcommand{\aob}[1]{"{\texttt{#1}}"}

\definecolor{lg}{rgb}{0.95,0.95,0.95}

\newcounter{htmlkey}
\setcounter{htmlkey}{2}
\usepackage{keystroke}
\newcommand{\sclxml}{\input{sclxml}}
\newcommand{\sglxml}{\input{sglxml}}

\newcommand{\mbf}[1]{<#1>}
\newcommand{\key}[1]{\red{#1}}

\newcommand{\isaacs}{I.S.A.A.C.S.}
\newcommand{\ISAACS}{Interactive Structure Analysis of Amorphous and Crystalline Systems}
\newcommand{\atomes}{{\em{\bf{atomes}}}}
\newcommand{\activp}{{\em{\bf{active}}}}
\newcommand{\rpc}{$R_C(n)$}
\newcommand{\rpn}{$R_N(n)$}
\newcommand{\pnr}{$P_N(n)$}
\newcommand{\pnrmin}{$P_{N_{\text{min}}}(n)$}
\newcommand{\pnrmax}{$P_{N_{\text{max}}}(n)$}
\newcommand{\pmin}{$P_{\text{min}}(n)$}
\newcommand{\pmax}{$P_{\text{max}}(n)$}
\newcommand{\smin}{$s_{min}$}
\newcommand{\smax}{$s_{max}$}

\newcommand{\ges}{GeS$_2$}
\newcommand{\sio}{SiO$_2$}
\newcommand{\nn}{rings with $n$ nodes}
\newcommand{\con}{connectivity}
\newcommand{\conp}{connectivity profile}
\newcommand{\rstat}{ring statistics}

\newcommand{\dlpoly}{\href{https://www.scd.stfc.ac.uk/Pages/DL\_POLY.aspx}{DL-POLY}}
\newcommand{\lammps}{\href{https://lammps.sandia.gov/}{LAMMPS}}
\newcommand{\cpmd}{\href{http://www.cpmd.org}{CPMD}}
\newcommand{\cptk}{\href{http://cp2k.berlios.de}{CP2K}}

\newcommand{\atomesweb}{https://atomes.ipcms.fr}

\renewcommand{\figurename}{Figure}
\renewcommand{\tablename}{Table}
\newcommand{\laf}{\\}
\newcommand{\image}[2]{\includegraphics*[width=#1cm, keepaspectratio=true, draft=\ddst]{#2}}
\newcommand{\jmage}[2]{\includegraphics*[height=#1cm, keepaspectratio=true, draft=\ddst]{#2}}

\newcommand{\myfigure}[5]{\begin{figure}[!#1]{\hypertarget{#2}{\begin{center}{#3\caption[#4]{#5}\label{#2}}\end{center}}}\end{figure}}
\newcommand{\mytable}[5]{\begin{table}[!#1]{\begin{center}{#3\caption[#4]{#5}\label{#2}}\end{center}}\end{table}}

\newsavebox{\cobox}
\def\script{
  \noindent \laf \laf
  \begin{lrbox}
  \cobox
  \begin{minipage}[l]{16cm}
  \begin{alltt}}
\def\endscript{
  \end{alltt}
  \end{minipage}
  \end{lrbox}
  \colorbox{lg}{\usebox{\cobox}}
  \vspace{0.125cm}\par\noindent}

\def\scri#1{
  \noindent \laf \laf
  \begin{lrbox}
  \cobox
  \begin{minipage}[l]{#1cm}
  \begin{alltt}}
\def\endscri{
  \end{alltt}
  \end{minipage}
  \end{lrbox}
  \colorbox{lg}{\usebox{\cobox}}
  \vspace{0.25cm}\par\noindent}

