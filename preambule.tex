\usepackage[utf8]{inputenc}
\usepackage[T1]{fontenc}
\usepackage{babel}

\usepackage{pifont}
% Les polices possibles (pour tout le document)
%\usepackage[bitstream-charter]{mathdesign}

%Font ps pour le pdf
\usepackage{pslatex}

%Hyperliens pour le pdf
\usepackage[pdfauthor   = {Sébastien\ Le\ Roux},
    pdftitle    = {atomes\ tutorials},
    pdfsubject = {atomes\ tutorials},
    pdfkeywords = {atomes\ tutorials\ guide\ help\ documentation},
    pdfcreator = {Latex+DviPDF},
    pdfproducer = {LaTeX+DviPDF},
    pdfstartview=FitV   % Ouverture avec ajustement de l'image
    dvips=true,         % Use hyperref with dvips
    colorlinks=true,    % Lien hypertext en couleur
    plainpages=false,   %
    pagebackref=true,   % Permet d'ajouter des liens retour dans la biblio ...
    backref=page,       % .. ces liens pointent vers les 'pages' des citations
    hyperindex=true,    % Ajoute des liens dans l'index
    linktocpage=true,   % Lien sur les numéros de page et non le text 
    breaklinks=true,    % Permet le retour à la ligne dans les liens trop longs
    urlcolor= blue,     % Couleur des liens externes
%    linkcolor= black,   % Couleur des liens internes
    bookmarks=true,     % Création des signets pour Acrobat
    bookmarksopen=false % Toute l'arborescence est dépliée à l'ouverture
]{hyperref}
%\usepackage[pageref]{backref}

%Insertion d'images
\usepackage{graphicx}
\DeclareGraphicsExtensions{.eps}

%Dessins latex
% Brouillon ? = afficher les images "false" ou seulement les cadres "true"
% effet sur tout le document
\newcommand{\ddst}{false}
%\usepackage{picins}

\usepackage{color}

% Mode verbatim avancé
\usepackage{alltt}

% Système de liste/énumération
\usepackage{pifont}
\usepackage{enumerate}

% Ecriture des mathématiques
\usepackage{amsmath}
\usepackage{amssymb}
\usepackage{amscd}
\usepackage{theorem}

\usepackage{pxfonts}

% tableaux
\usepackage{hhline}
%\usepackage{array}
\usepackage{multirow}
\usepackage{tabls}

%% Mise en page
\voffset         0.0cm
\hoffset         0.0cm
\textheight     24.0cm
\textwidth      16.0cm
\topmargin      -0.5cm
\oddsidemargin   0.0cm
\evensidemargin  0.0cm

% pages en landscape dans document portrait
\usepackage{lscape}

% Aspect de pages
\usepackage{setspace}
%\onehalfspacing
%\doublespacing
%\setstretch{3}

%\usepackage{fancyhdr,fancybox}
%\fancyhead{}
%\fancyhead[RO]{\scriptsize{\slshape\rightmark}}
%\fancyhead[LO]{\scriptsize{\thepage}}
%\fancyhead[RE]{\scriptsize{\thepage}}
%\fancyhead[LE]{\scriptsize{\slshape\leftmark}}
%\pagestyle{fancy}

% Numérotation pour les sous-sous-sections
\setcounter{secnumdepth}{3}
% Table des matières/figures/tables par chapitre
%\usepackage{minitoc}
% Pour placer des notes de bas de pages dans les titres
%\usepackage[stable]{footmisc}

%Définitions de théorèmes
\theoremstyle{plain}
\theoremheaderfont{\scshape}
\theorembodyfont{\normalfont\itshape}
\newtheorem{HK}{Théorème}

% Créer un environnement résumé
\def\abstract{
   \begin{center}
   \begin{minipage}{12cm}
   \begin{center}{\bf Résumé}\end{center}\par\small}
\def\endabstract{\par\end{minipage}\end{center}\vspace{1cm}}

% Bilblio:
\input{bibsymb}
%\usepackage{chapterbib}

%\usepackage[square,comma,sort&compress]{natbib}
%\usepackage{hypernat}

% Algo XML
\usepackage{listings}

% Texte souligné
\usepackage[normalem]{ulem}

% Mise en forme des légendes
\usepackage[hang]{caption2}
%\usepackage[hang]{caption}
\renewcommand{\captionfont}{\it}
\renewcommand{\captionlabelfont}{\bf}
\renewcommand{\captionlabeldelim}{$\quad$}

%\usepackage[format=plain,labelfont=bf,up,textfont=it,up]{caption}
\newcommand{\red}[1]{\textcolor{red}{#1}}
\newcommand{\blue}[1]{\textcolor{blue}{#1}}
\newcommand{\green}[1]{\textcolor{green}{#1}}
\newcommand{\violet}[1]{\textcolor{violet}{#1}}
\newcommand{\pink}[1]{\textcolor{pink}{#1}}
\newcommand{\aob}[1]{"{\texttt{#1}}"}

\definecolor{lg}{rgb}{0.95,0.95,0.95}

\newcounter{htmlkey}
\setcounter{htmlkey}{2}
\usepackage{keystroke}
\newcommand{\sclxml}{\input{sclxml}}
\newcommand{\sglxml}{\input{sglxml}}

\newcommand{\mbf}[1]{<#1>}
\newcommand{\key}[1]{\red{#1}}

\newcommand{\isaacs}{I.S.A.A.C.S.}
\newcommand{\ISAACS}{Interactive Structure Analysis of Amorphous and Crystalline Systems}
\newcommand{\atomes}{{\em{\bf{atomes}}}}
\newcommand{\activp}{{\em{\bf{active}}}}
\newcommand{\rpc}{$R_C(n)$}
\newcommand{\rpn}{$R_N(n)$}
\newcommand{\pnr}{$P_N(n)$}
\newcommand{\pnrmin}{$P_{N_{\text{min}}}(n)$}
\newcommand{\pnrmax}{$P_{N_{\text{max}}}(n)$}
\newcommand{\pmin}{$P_{\text{min}}(n)$}
\newcommand{\pmax}{$P_{\text{max}}(n)$}
\newcommand{\smin}{$s_{min}$}
\newcommand{\smax}{$s_{max}$}

\newcommand{\ges}{GeS$_2$}
\newcommand{\sio}{SiO$_2$}
\newcommand{\nn}{rings with $n$ nodes}
\newcommand{\con}{connectivity}
\newcommand{\conp}{connectivity profile}
\newcommand{\rstat}{ring statistics}

\newcommand{\dlpoly}{\href{https://www.scd.stfc.ac.uk/Pages/DL\_POLY.aspx}{DL-POLY}}
\newcommand{\lammps}{\href{https://lammps.sandia.gov/}{LAMMPS}}
\newcommand{\cpmd}{\href{http://www.cpmd.org}{CPMD}}
\newcommand{\cptk}{\href{http://cp2k.berlios.de}{CP2K}}

\newcommand{\atomesweb}{https://atomes.ipcms.fr}

\renewcommand{\figurename}{Figure}
\renewcommand{\tablename}{Table}
\newcommand{\laf}{\\}
\newcommand{\image}[2]{\includegraphics*[width=#1cm, keepaspectratio=true, draft=\ddst]{#2}}
\newcommand{\jmage}[2]{\includegraphics*[height=#1cm, keepaspectratio=true, draft=\ddst]{#2}}

\newcommand{\myfigure}[5]{\begin{figure}[!#1]{\hypertarget{#2}{\begin{center}{#3\caption[#4]{#5}\label{#2}}\end{center}}}\end{figure}}
\newcommand{\mytable}[5]{\begin{table}[!#1]{\begin{center}{#3\caption[#4]{#5}\label{#2}}\end{center}}\end{table}}

\newsavebox{\cobox}
\def\script{
  \noindent \laf \laf
  \begin{lrbox}
  \cobox
  \begin{minipage}[l]{16cm}
  \begin{alltt}}
\def\endscript{
  \end{alltt}
  \end{minipage}
  \end{lrbox}
  \colorbox{lg}{\usebox{\cobox}}
  \vspace{0.125cm}\par\noindent}

\def\scri#1{
  \noindent \laf \laf
  \begin{lrbox}
  \cobox
  \begin{minipage}[l]{#1cm}
  \begin{alltt}}
\def\endscri{
  \end{alltt}
  \end{minipage}
  \end{lrbox}
  \colorbox{lg}{\usebox{\cobox}}
  \vspace{0.25cm}\par\noindent}


\newcommand{\toolgrfig}{
\begin{tabular}{cp{0.5cm}cp{0.5cm}c}
\hspace{-1cm}\image{5}{img/tuto-9/tb-001} & \raisebox{2.0cm}{$\Longrightarrow$} &
\image{5}{img/tuto-9/tb-001} & \raisebox{2.0cm}{$\Longrightarrow$} &
\image{5}{img/tuto-9/tb-003} \\
\end{tabular}
\begin{tabular}{cc}
\image{12}{img/tuto-9/gr} & \raisebox{10cm}{$\Swarrow$}
\end{tabular}}

\newcommand{\mapenvfig}{
\begin{tabular}{cp{0.5cm}c}
\hspace{-1cm}\image{9}{img/tuto-9/map-env} & \raisebox{2.0cm}{$\Longrightarrow$} & \image{5}{img/tuto-9/gSiO2-003}
\end{tabular}}

\newcommand{\editnewfig}{
\begin{tabular}{cp{0.5cm}c}
\hspace{-1cm}\image{8}{img/tuto-9/edit\_as\_new} & \raisebox{2.0cm}{$\Longrightarrow$} & \image{8}{img/tuto-9/new\_proj}
\end{tabular}}

\newcommand{\addfinalfig}{
\begin{tabular}{cp{0.5cm}c}
\hspace{-1cm}\jmage{6}{img/tuto-9/add\_data} & \raisebox{3.0cm}{$\Longrightarrow$} & \jmage{7}{img/tuto-9/final}
\end{tabular}}


