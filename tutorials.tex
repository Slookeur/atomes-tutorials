\documentclass{these-seb}

\usepackage[utf8]{inputenc}
\usepackage[T1]{fontenc}
\usepackage{babel}

\usepackage{pifont}
% Les polices possibles (pour tout le document)
%\usepackage[bitstream-charter]{mathdesign}

%Font ps pour le pdf
\usepackage{pslatex}

%Hyperliens pour le pdf
\usepackage[pdfauthor   = {Sébastien\ Le\ Roux},
    pdftitle    = {atomes\ tutorials},
    pdfsubject = {atomes\ tutorials},
    pdfkeywords = {atomes\ tutorials\ guide\ help\ documentation},
    pdfcreator = {Latex+DviPDF},
    pdfproducer = {LaTeX+DviPDF},
    pdfstartview=FitV   % Ouverture avec ajustement de l'image
    dvips=true,         % Use hyperref with dvips
    colorlinks=true,    % Lien hypertext en couleur
    plainpages=false,   %
    pagebackref=true,   % Permet d'ajouter des liens retour dans la biblio ...
    backref=page,       % .. ces liens pointent vers les 'pages' des citations
    hyperindex=true,    % Ajoute des liens dans l'index
    linktocpage=true,   % Lien sur les numéros de page et non le text 
    breaklinks=true,    % Permet le retour à la ligne dans les liens trop longs
    urlcolor= blue,     % Couleur des liens externes
%    linkcolor= black,   % Couleur des liens internes
    bookmarks=true,     % Création des signets pour Acrobat
    bookmarksopen=false % Toute l'arborescence est dépliée à l'ouverture
]{hyperref}
%\usepackage[pageref]{backref}

%Insertion d'images
\usepackage{graphicx}
\DeclareGraphicsExtensions{.eps}

%Dessins latex
% Brouillon ? = afficher les images "false" ou seulement les cadres "true"
% effet sur tout le document
\newcommand{\ddst}{false}
%\usepackage{picins}

\usepackage{color}

% Mode verbatim avancé
\usepackage{alltt}

% Système de liste/énumération
\usepackage{pifont}
\usepackage{enumerate}

% Ecriture des mathématiques
\usepackage{amsmath}
\usepackage{amssymb}
\usepackage{amscd}
\usepackage{theorem}

\usepackage{pxfonts}

% tableaux
\usepackage{hhline}
%\usepackage{array}
\usepackage{multirow}
\usepackage{tabls}

%% Mise en page
\voffset         0.0cm
\hoffset         0.0cm
\textheight     24.0cm
\textwidth      16.0cm
\topmargin      -0.5cm
\oddsidemargin   0.0cm
\evensidemargin  0.0cm

% pages en landscape dans document portrait
\usepackage{lscape}

% Aspect de pages
\usepackage{setspace}
%\onehalfspacing
%\doublespacing
%\setstretch{3}

%\usepackage{fancyhdr,fancybox}
%\fancyhead{}
%\fancyhead[RO]{\scriptsize{\slshape\rightmark}}
%\fancyhead[LO]{\scriptsize{\thepage}}
%\fancyhead[RE]{\scriptsize{\thepage}}
%\fancyhead[LE]{\scriptsize{\slshape\leftmark}}
%\pagestyle{fancy}

% Numérotation pour les sous-sous-sections
\setcounter{secnumdepth}{3}
% Table des matières/figures/tables par chapitre
%\usepackage{minitoc}
% Pour placer des notes de bas de pages dans les titres
%\usepackage[stable]{footmisc}

%Définitions de théorèmes
\theoremstyle{plain}
\theoremheaderfont{\scshape}
\theorembodyfont{\normalfont\itshape}
\newtheorem{HK}{Théorème}

% Créer un environnement résumé
\def\abstract{
   \begin{center}
   \begin{minipage}{12cm}
   \begin{center}{\bf Résumé}\end{center}\par\small}
\def\endabstract{\par\end{minipage}\end{center}\vspace{1cm}}

% Bilblio:
% Bilblio:
\newcommand{\aap}{Astron. \& Astrophys.}
\newcommand{\aasup}{Astron. \& Astrophys. Suppl. Ser.}
\newcommand{\aj}{Astron. J.}
\newcommand{\aph}{Acta Phys.}
\newcommand{\act}{Acta Cryst.}
\newcommand{\actc}{Acta Cryst.}
\newcommand{\acta}{Acta Cryst. A}
\newcommand{\actb}{Acta Cryst. B}
\newcommand{\advp}{Adv. Phys.}
\newcommand{\ajp}{Amer. J. Phys.}
\newcommand{\ajm}{Amer. J. Math.}
\newcommand{\amsci}{Amer. Sci.}
\newcommand{\anofd}{Ann. Fluid Dyn.}
\newcommand{\am}{Ann. Math.}
\newcommand{\ap}{Ann. Phys. (NY)}
\newcommand{\adp}{Ann. Phys. (Leipzig)}
\newcommand{\ao}{Appl. Opt.}
\newcommand{\apl}{Appl. Phys. Lett.}
\newcommand{\app}{Astroparticle Phys.}
\newcommand{\apj}{Astrophys. J.}
\newcommand{\apjsup}{Astrophys. J. Suppl.}
\newcommand{\apss}{Astrophys. Space Sci.}
\newcommand{\araa}{Ann. Rev. Astron. Astrophys.}
\newcommand{\baas}{Bull. Amer. Astron. Soc.}
\newcommand{\baps}{Bull. Amer. Phys. Soc.}
\newcommand{\cmp}{Comm. Math. Phys.}
\newcommand{\cpam}{Commun. Pure Appl. Math.}
\newcommand{\cppcf}{Comm. Plasma Phys. \& Controlled Fusion}
\newcommand{\cpc}{Comp. Phys. Comm.}
\newcommand{\cms}{Comp. Mat. Sci.}
\newcommand{\cqg}{Class. Quant. Grav.}
\newcommand{\cra}{C. R. Acad. Sci. A}
\newcommand{\crv}{Chem. Rev.}
\newcommand{\cp}{Chem. Phys.}
\newcommand{\cma}{Chem. Mater.}
\newcommand{\epl}{Eur. Phys. Lett.}
\newcommand{\ejm}{Eur. J. Mineral.}
\newcommand{\fed}{Fusion Eng. \& Design}
\newcommand{\ft}{Fusion Tech.}
\newcommand{\grg}{Gen. Relativ. Gravit.}
\newcommand{\ieeens}{IEEE Trans. Nucl. Sci.}
\newcommand{\ieeeps}{IEEE Trans. Plasma Sci.}
\newcommand{\ijimw}{Interntl. J. Infrared \& Millimeter Waves}
\newcommand{\ip}{Infrared Phys.}
\newcommand{\irp}{Infrared Phys.}
\newcommand{\jap}{J. Appl. Phys.}
\newcommand{\jac}{J. Appl. Cryst.}
\newcommand{\jacs}{J. Am. Chem. Soc.}
\newcommand{\jamcs}{J. Am. Ceram. Soc.}
\newcommand{\jasa}{J. Acoust. Soc. America}
\newcommand{\jcp}{J. Chem. Phys.}
\newcommand{\jcop}{J. Comp. Phys.}
\newcommand{\jetp}{Sov. Phys.--JETP}
\newcommand{\jetpl}{JETP Lett.}
\newcommand{\jfe}{J. Fusion Energy}
\newcommand{\jfm}{J. Fluid Mech.}
\newcommand{\jmp}{J. Math. Phys.}
\newcommand{\jne}{J. Nucl. Energy}
\newcommand{\jnec}{J. Nucl. Energy, C: Plasma Phys., Accelerators, Thermonucl. Res.}
\newcommand{\jncs}{J. Non-Cryst. Solids.}
\newcommand{\jnm}{J. Nucl. Mat.}
\newcommand{\joam}{J. Optoelect. Adv. Mat.}
\newcommand{\jpcssp}{J. Phys. C: Solid State Phys.}
\newcommand{\jpc}{J. Phys. Chem.}
\newcommand{\jpcs}{J. Phys. Chem. Sol.}
\newcommand{\jpcm}{J. Phys.: Cond. Mat.}
\newcommand{\jpp}{J. Plasma Phys.}
\newcommand{\jpsj}{J. Phys. Soc. Japan}
\newcommand{\jqc}{J. Quant. Chem.}
\newcommand{\jssc}{J. Sol. Stat. Chem.}
\newcommand{\jsi}{J. Sci. Instrum.}
\newcommand{\jvst}{J. Vac. Sci. \& Tech.}
\newcommand{\mcp}{Mat. Chem. Phys.}
\newcommand{\nat}{Nature}
\newcommand{\nature}{Nature}
\newcommand{\nedf}{Nucl. Eng. \& Design/Fusion}
\newcommand{\nf}{Nucl. Fusion}
\newcommand{\nim}{Nucl. Inst. \& Meth.}
\newcommand{\nimpr}{Nucl. Inst. \& Meth. in Phys. Res.}
\newcommand{\np}{Nucl. Phys.}
\newcommand{\npb}{Nucl. Phys. B}
\newcommand{\ntf}{Nucl. Tech./Fusion}
\newcommand{\npbpc}{Nucl. Phys. B (Proc. Suppl.)}
\newcommand{\inc}{Nuovo Cimento}
\newcommand{\nc}{Nuovo Cimento}
\newcommand{\pcg}{Phys. Chem. Glasses}
\newcommand{\pf}{Phys. Fluids}
\newcommand{\pfa}{Phys. Fluids A: Fluid Dyn.}
\newcommand{\pfb}{Phys. Fluids B: Plasma Phys.}
\newcommand{\pl}{Phys. Lett.}
\newcommand{\pla}{Phys. Lett. A}
\newcommand{\plb}{Phys. Lett. B}
\newcommand{\prep}{Phys. Rep.}
\newcommand{\pnas}{Proc. Nat. Acad. Sci. USA}
\newcommand{\pp}{Phys. Plasmas}
\newcommand{\ppcf}{Plasma Phys. \& Controlled Fusion}
\newcommand{\phitrsl}{Philos. Trans. Roy. Soc. London}
\newcommand{\plmb}{Phil. Mag. B} 
\newcommand{\pml}{Phil. Mag. Lett.}
\newcommand{\pmm}{Phil. Mag.}
\newcommand{\prl}{Phys. Rev. Lett.}
\newcommand{\pr}{Phys. Rev.}
\newcommand{\physrev}{Phys. Rev.}
\newcommand{\pra}{Phys. Rev. A}
\newcommand{\prb}{Phys. Rev. B}
\newcommand{\prc}{Phys. Rev. C}
\newcommand{\prd}{Phys. Rev. D}
\newcommand{\pre}{Phys. Rev. E}
\newcommand{\ps}{Phys. Scripta}
\newcommand{\pstb}{Phys. Stat. Sol. b}
\newcommand{\procrsl}{Proc. Roy. Soc. London}
\newcommand{\rmp}{Rev. Mod. Phys.}
\newcommand{\rsi}{Rev. Sci. Inst.}
\newcommand{\rpp}{Rep. Prog. Phys.}
\newcommand{\science}{Science}
\newcommand{\sciam}{Sci. Am.}
\newcommand{\susc}{Surf. Sci}
\newcommand{\sam}{Stud. Appl. Math.}
\newcommand{\sjpp}{Sov. J. Plasma Phys.}
\newcommand{\spd}{Sov. Phys.--Doklady}
\newcommand{\sptp}{Sov. Phys.--Tech. Phys.}
\newcommand{\spu}{Sov. Phys.--Uspeki}
\newcommand{\skt}{Sky and Telesc.}
\newcommand{\ssi}{Solid State Ionics}
\newcommand{\ssc}{Solid State Com.}
\newcommand{\ssnmr}{Solid State Nuc. Mag. Res.}
\newcommand{\zfp}{Zs. f. Phys.}
\newcommand{\zk}{Z. Kristallogr.}


%\usepackage{chapterbib}

%\usepackage[square,comma,sort&compress]{natbib}
%\usepackage{hypernat}

% Algo XML
\usepackage{listings}

% Texte souligné
\usepackage[normalem]{ulem}

% Mise en forme des légendes
\usepackage[hang]{caption2}
%\usepackage[hang]{caption}
\renewcommand{\captionfont}{\it}
\renewcommand{\captionlabelfont}{\bf}
\renewcommand{\captionlabeldelim}{$\quad$}

%\usepackage[format=plain,labelfont=bf,up,textfont=it,up]{caption}
\newcommand{\red}[1]{\textcolor{red}{#1}}
\newcommand{\blue}[1]{\textcolor{blue}{#1}}
\newcommand{\green}[1]{\textcolor{green}{#1}}
\newcommand{\violet}[1]{\textcolor{violet}{#1}}
\newcommand{\pink}[1]{\textcolor{pink}{#1}}
\newcommand{\aob}[1]{"{\texttt{#1}}"}

\definecolor{lg}{rgb}{0.95,0.95,0.95}

\newcounter{htmlkey}
\setcounter{htmlkey}{2}
\usepackage{keystroke}
\newcommand{\sclxml}{\input{sclxml}}
\newcommand{\sglxml}{\input{sglxml}}

\newcommand{\mbf}[1]{<#1>}
\newcommand{\key}[1]{\red{#1}}

\newcommand{\isaacs}{I.S.A.A.C.S.}
\newcommand{\ISAACS}{Interactive Structure Analysis of Amorphous and Crystalline Systems}
\newcommand{\atomes}{{\em{\bf{atomes}}}}
\newcommand{\activp}{{\em{\bf{active}}}}
\newcommand{\rpc}{$R_C(n)$}
\newcommand{\rpn}{$R_N(n)$}
\newcommand{\pnr}{$P_N(n)$}
\newcommand{\pnrmin}{$P_{N_{\text{min}}}(n)$}
\newcommand{\pnrmax}{$P_{N_{\text{max}}}(n)$}
\newcommand{\pmin}{$P_{\text{min}}(n)$}
\newcommand{\pmax}{$P_{\text{max}}(n)$}
\newcommand{\smin}{$s_{min}$}
\newcommand{\smax}{$s_{max}$}

\newcommand{\ges}{GeS$_2$}
\newcommand{\sio}{SiO$_2$}
\newcommand{\nn}{rings with $n$ nodes}
\newcommand{\con}{connectivity}
\newcommand{\conp}{connectivity profile}
\newcommand{\rstat}{ring statistics}

\newcommand{\dlpoly}{\href{https://www.scd.stfc.ac.uk/Pages/DL\_POLY.aspx}{DL-POLY}}
\newcommand{\lammps}{\href{https://lammps.sandia.gov/}{LAMMPS}}
\newcommand{\cpmd}{\href{http://www.cpmd.org}{CPMD}}
\newcommand{\cptk}{\href{http://cp2k.berlios.de}{CP2K}}

\newcommand{\atomesweb}{https://atomes.ipcms.fr}

\renewcommand{\figurename}{Figure}
\renewcommand{\tablename}{Table}
\newcommand{\laf}{\\}
\newcommand{\image}[2]{\includegraphics*[width=#1cm, keepaspectratio=true, draft=\ddst]{#2}}
\newcommand{\jmage}[2]{\includegraphics*[height=#1cm, keepaspectratio=true, draft=\ddst]{#2}}

\newcommand{\myfigure}[5]{\begin{figure}[!#1]{\hypertarget{#2}{\begin{center}{#3\caption[#4]{#5}\label{#2}}\end{center}}}\end{figure}}
\newcommand{\mytable}[5]{\begin{table}[!#1]{\begin{center}{#3\caption[#4]{#5}\label{#2}}\end{center}}\end{table}}

\newsavebox{\cobox}
\def\script{
  \noindent \laf \laf
  \begin{lrbox}
  \cobox
  \begin{minipage}[l]{16cm}
  \begin{alltt}}
\def\endscript{
  \end{alltt}
  \end{minipage}
  \end{lrbox}
  \colorbox{lg}{\usebox{\cobox}}
  \vspace{0.125cm}\par\noindent}

\def\scri#1{
  \noindent \laf \laf
  \begin{lrbox}
  \cobox
  \begin{minipage}[l]{#1cm}
  \begin{alltt}}
\def\endscri{
  \end{alltt}
  \end{minipage}
  \end{lrbox}
  \colorbox{lg}{\usebox{\cobox}}
  \vspace{0.25cm}\par\noindent}


\newcommand{\toolgrfig}{
\begin{tabular}{cp{0.5cm}cp{0.5cm}c}
\hspace{-1cm}\image{5}{img/tuto-9/tb-001} & \raisebox{2.0cm}{$\Longrightarrow$} &
\image{5}{img/tuto-9/tb-001} & \raisebox{2.0cm}{$\Longrightarrow$} &
\image{5}{img/tuto-9/tb-003} \\
\end{tabular}
\begin{tabular}{cc}
\image{12}{img/tuto-9/gr} & \raisebox{10cm}{$\Swarrow$}
\end{tabular}}

\newcommand{\mapenvfig}{
\begin{tabular}{cp{0.5cm}c}
\hspace{-1cm}\image{9}{img/tuto-9/map-env} & \raisebox{2.0cm}{$\Longrightarrow$} & \image{5}{img/tuto-9/gSiO2-003}
\end{tabular}}

\newcommand{\editnewfig}{
\begin{tabular}{cp{0.5cm}c}
\hspace{-1cm}\image{8}{img/tuto-9/edit\_as\_new} & \raisebox{2.0cm}{$\Longrightarrow$} & \image{8}{img/tuto-9/new\_proj}
\end{tabular}}

\newcommand{\addfinalfig}{
\begin{tabular}{cp{0.5cm}c}
\hspace{-1cm}\jmage{6}{img/tuto-9/add\_data} & \raisebox{3.0cm}{$\Longrightarrow$} & \jmage{7}{img/tuto-9/final}
\end{tabular}}




\includeonly{chap-1/chap-1,chap-2/chap-2,chap-3/chap-3}

\begin{document}

% Préparation pour mini-table des matières, des figures et des tables si besoin
%\dominitoc
%\dominilof
%\dominilot

%
% Atomes Users manual	
%
\titleEN{\hspace{-1cm}\includegraphics*[width=17cm, keepaspectratio=true, draft=\ddst]{img/overview4}\\ \bigskip \atomes\ tutorials}

\author{Sébastien {\sc{Le Roux}}\quad\href{mailo:sebastien.leroux@ipcms.unistra.fr}{sebastien.leroux@ipcms.unistra.fr}}
\address{sebastien.leroux@ipcms.unistra.fr}
\universite{Institut de Physique et Chimie des Matériaux de Strasbourg\\
Département des Matériaux Organiques\
BP 43, 23 rue du Loess,\\
F-670234 Strasbourg Cedex 2, France}

\beforepreface

\setcounter{page}{1}            % Je recommence la numerotation a un pour pas
                                % decaler l'alternance droite/gauche et que
                                % les pages impaires soient bien a droite.
%\setcounter{tocdepth}{4}

% Mise en forme TOC/TOF/LOT
{\small \tableofcontents}
\addcontentsline{toc}{chapter}{Contents}

\afterpreface

\vfill                          % Je remplis avec du rien (\vfill}
\pagebreak                      % Je change de page.

%{\small \printindex}

% Intro
\chapter{Introduction}
\label{introtuto}

The following regroups tutorials to help user discover the \atomes\ program. \\
Tutorials cover many aspects of \atomes\ capabilities:
\begin{itemize}
\item Importing atomic coordinates - see [\ref{tuto-1}]
\item Opening \atomes\ project and workspace file(s) - see [\ref{tuto-2}]
\item Evaluating and comparing physico-chemical properties - see [\ref{tuto-3}]
\item Visualizing clone(s) - see [\ref{tuto-4}]
\item Visualizing coordination polyhedra - see [\ref{tuto-5}]
\item Using the cell edition utility - see [\ref{tuto-6}]
\item Using the model edition utility - see [\ref{tuto-7}]
\item Using the crystal builder - see [\ref{tuto-8}]
\item Getting information on selected structural features: g(r) in {\it{g}}-SiO$_2$ - see [\ref{tuto-9}]
\item Building a crystal, then creating and passivating a surface - see [\ref{tuto-10}]
\end{itemize}
 

\chapter{ Getting information on selected structural features: g(r) in $g$-SiO$_2$}
\label{tuto-1}

The following example illustrates how \atomes\ can be use to retrieve information on selected structural features:

\section{Open the \atomes\ $g$-SiO$_2$ project file}

Open the \atomes\ project file for the system "\href{http://atomes.ipcms.fr/wp-content/uploads/2022/08/gSiO2-custom-rings.tar.bz2}{$g$-SiO$_2$ with ring statistics and custom color map, 3000 atoms}" \\[0.25cm]
\image{14}{img/tuto-1/gSiO2-001}\\
Note that the "\texttt{Toolboxes}" dialog is active and some results, from calculation performed previously and stored in the project file, 
are available, including for the radial distribution function g(r). 

\clearpage

\section{Open the "Environment configuration" window}

Go back to the 3D window of the $g$-SiO$_2$ model, 
for clarity purpose change color map to standard chemical species: from the "\texttt{OpenGL~->~Color~Scheme(s)}" menu, or press \keystroke{a} as many times as required. 
Then open the "\texttt{Environments configuration}" window: from the "\texttt{Chemistry}" menu or press \Ctrl+\keystroke{e}, and open the tab labelled "\texttt{King's~ring(s){\bf{[KR]}}}". \\[0.5cm]
\begin{tabular}{cp{0.5cm}c}
\hspace{-1cm}\image{9}{img/tuto-1/map-env} & \raisebox{2.0cm}{$\Longrightarrow$} & \image{5}{img/tuto-1/gSiO2-003}
\end{tabular} \\[0.5cm]
This tab provides interactors with King's ring statistics calculation results: color of the ring's polyhedra, and then options to show/hide, label/unlabel, 
or select/unselect atom(s) involved in any ring(s) of the corresponding size.

\clearpage

\section{Select all atoms in ring(s) of size 6}

"\texttt{Pick}" all atoms in ring(s) of size 6: atoms in the 3D window are then selected and highlighted in light blue color. \\[0.5cm]
\image{15}{img/tuto-1/sel}

\clearpage

\section{Create a new model using the atom selection} 

Go back to the 3D window, and mouse right click on any selected atom, then browse the contextual menu that pops up to "\texttt{Edit~as~New~Project}", 
then select "\texttt{All~Selected~Atom(s)/Bond(s)}" \\[0.5cm]
\begin{tabular}{cp{0.5cm}c}
\hspace{-1cm}\image{8}{img/tuto-1/edit\_as\_new} & \raisebox{2.0cm}{$\Longrightarrow$} & \image{8}{img/tuto-1/new\_proj}
\end{tabular}
\\[0.5cm]
This will create a new \atomes\ project that will immediately be inserted in the program's workspace, and the corresponding 3D window will appear. \\ 
This new model contains all atoms involved in King's rings of size 6 selected previously in the disordered SiO$_2$ model. 

\clearpage

\section{Correct the periodicity of the new model} 

From the main interface open the "\texttt{Edit}" menu and select "\texttt{Box~and~Periodicity}" 
to reproduce the periodicity of the initial model describing properly the simulation box (a~$=$~b~$=$~c~$= 35.6621$~\AA, $\alpha = \beta = \gamma = 90~^{\circ}$, and be sure to use Periodic Boundary Conditions), then apply changes. \\[0.5cm]
\image{15}{img/tuto-1/perio} \\[0.5cm]
The new model now uses periodic boundary conditions, if required the box can be displayed in the 3D window using the corresponding menu button: "\texttt{Model~->~Box~->~Show/Hide}".

\clearpage

\section{Compute the g(r) for the new model}

Again from the main interface open the "\texttt{Analyze}" menu, then select "\texttt{g(r)~/~G(r)}", to open the associated calculation dialog. 
Set parameter(s) to acceptable value(s) and run the analysis. \\[0.5cm] 
\image{15}{img/tuto-1/run-gr}

\clearpage

\section{Visualize the results of the calculation} 

When the calculation is completed, close the dialog, and note that the content of the "\texttt{Toolboxes}" dialog 
is updated, the results of the calculation being available:\\[0.5cm]
\begin{tabular}{cp{0.5cm}cp{0.5cm}c}
\hspace{-1cm}\image{5}{img/tuto-1/tb-001} & \raisebox{2.0cm}{$\Longrightarrow$} &
\image{5}{img/tuto-1/tb-002} & \raisebox{2.0cm}{$\Longrightarrow$} &
\image{5}{img/tuto-1/tb-003} \\
\end{tabular}
\begin{tabular}{cc}
\hspace{2cm}\image{10}{img/tuto-1/gr} & \raisebox{8cm}{$\Swarrow$}
\end{tabular}
\\[0.5cm]
In the "\texttt{Toolboxes}" dialog open the result of the calculation to display the g(r) of the active project, in this case the new model created from the atom selection. 

\clearpage

\section{Add the calculation data for the entire $g$-SiO$_2$ model}

On the curve window press the mouse right click, browse the menu to "\texttt{Add~Data~Set~->~$g$SiO}$_2$\texttt{~->~g(r)~neutrons}" 
to add the corresponding data set to the representation: \\[0.5cm]
\begin{tabular}{cp{0.5cm}c}
\hspace{-1cm}\jmage{6}{img/tuto-1/add\_data} & \raisebox{3.0cm}{$\Longrightarrow$} & \jmage{7}{img/tuto-1/final}
\end{tabular}
\\[0.5cm]
Note that projects opened in \atomes\ share data. Therefore calculation results, from any model in the workspace and provided that the calculations have been performed, can be compared easily. 
In this example it allows to compare the g(r) for the $g$-SiO$_2$ system with a partial contribution from the atoms involved in King's rings of size 6. \\
Comprehensive layout options are available and presentation can be clarified, each data set customized, so that the results of the analysis are quickly publication ready.

\chapter{Building a crystal, then creating and passivating a surface}
\label{tuto-2}

\section{Create an empty project to build a crystal}

Create a new empty project: "\texttt{Workspace~->~Project~->~New}" or press \Ctrl+\keystroke{n}, then open the crystal builder for the new, and empty, 3D window: "\texttt{Tools~->~Edit~->~Crystal~Builder}" \\[0.5cm]
\image{14}{img/tuto-2/002-cbuild}

\clearpage

\section{Build a large diamond-like structure}

Adjust parameters to selected space group and corresponding settings, in this example large diamond-like crystal: Cubic, Face-centered, space group N°227 "Fd-3m (1)" \\ 
Select the object(s) to be inserted at crystalline position(s), in this example C atom(s), note that in \atomes\ it is also possible to use molecular fragments inserted from the software library or from any project opened in the workspace. 
If needed adjust position, and / or occupancy, then remember to select the object(s) to be actually inserted when building the crystal by clicking on "Insert". 
Then simply click on "\texttt{Build}". \\[0.5cm]
\image{14}{img/tuto-2/003-building}
\\[0.5cm]
The 3D window is populated with the newly created structure, in this example a "20$\times$20$\times$20" supercell of pure diamond. 

\clearpage

\section{Open the surface creation utility}

To create a surface in this crystalline material, open the dedicated tool: "\texttt{Tools~->~Edit~->~Cell~->~Cut~Slab}" \\[0.5cm]
\image{14}{img/tuto-2/004-slab}

\clearpage

\section{Prepare and cut a cylinder slab}

Adjust the shape of the slab to "\texttt{Cylinder}", in the "\texttt{Size}" options adjust the cylinder length and radius to 80~\AA\ and 15~\AA\ respectively. 
You can display the slab using the "\texttt{Show/Hide~slab}" interactor, also export the results of the cutting/passivating in a new project by activating the corresponding option. \\
Check on "\texttt{Passivate surface}" and, in the option dialog that immediately opens, and as illustrated select to passivate empty C-C bonds created when cutting the slab with water molecule. \\[0.5cm]
\image{16}{img/tuto-2/005-cutting}
\\[0.5cm]
Then just click on "\texttt{Cut~this~slab~now~!}" to create a new project, that will contain the new material: a C diamond cell where the atoms inside the cylinder are removed, creating a surface, and where non-bridging bonds have been passivated using H$_2$O molecules.

\clearpage

\section{Check the model and open the "Model Edition" dialog}

Newly inserted water molecules, under selection, appear in light blue. If you rotate to model box to face right, water molecules appear perfectly aligned. 
To create disorder open the "\texttt{Random~Move}" tab of the "\texttt{Model~Edition}" dialog: "\texttt{Tools~->~Edit~->~Atoms~->~Random~Move}"\\[0.5cm] 
\image{15}{img/tuto-2/006-roran}

\clearpage

\section{Randomly rotate all water molecules to create disorder}

This interface allows to search for, and move randomly, atom(s) or group of atom(s), selection target(s) appearing in the table bellow. 
Also note that, quite conveniently, "\texttt{All~selected~atoms}" are by default targets of the action(s) to be performed. \\
To work on water molecules set the search parameters as follow:
\begin{itemize}
\item "\texttt{For}": use "\texttt{Group~of~atomes:~all}" to work on group of atoms, and not atom(s) individually, and to display a single interactor for all object(s) instead of a list detailed list. 
\item "\texttt{Filter}": use "\texttt{Partial~coordination}", to display a list of partial coordination(s) matching previously set criteria. 
\end{itemize}
Then selected to "\texttt{Rotate}" all "\texttt{O[H$_2$]}" coordination spheres, that is all water molecules, set a maximum Mean Square Displacement by adjusting the corresponding parameter, and if required specify how many times the process is to be repeated. \\[0.5cm]
\image{15}{img/tuto-2/007-random}\\
Then simply click on "\texttt{Move atom(s)}" to rotate randomly and separately each water molecule in the model.



%%%%%%%%%%%%%%%%%%%%%%%%%%%%%% Biblio %%%%%%%%%%%%%%%%%%%%%%%%%%%%%%
%\bibliographystyle{these}
%\bibliography{full-biblio}

% Colophon
\colophon{This document has been prepared using the Linux operating system and Free softwares:
\begin{tabular}{lc}
The text editor & "\href{http://www.vim.org/}{gVim}" \\
The GNU image manipulation program & "\href{http://www.gimp.org/}{The Gimp}" \\
The WYSIWYG plotting tool & "\href{http://plasma-gate.weizmann.ac.il/Grace/}{Grace}" \\
And the document preparation system & "\href{http://www.latex-project.org/}{\LaTeXe}".
\end{tabular}}

\end{document}

